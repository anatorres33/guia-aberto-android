% -- Preparando o Ambiente de Desenvolvimento
\chapter{Preparando o Ambiente de Desenvolvimento}

\section{Introdução}

O desenvolvimento de aplicativos para a plataforma Android é feito na linguagem Java.
Para esta apostila serão utilizados os seguintes aplicativos e bibliotecas:

% -- TODO: procurar como configurar o mesmo bullet pra todos os itens
\begin{itemize}
	\item Ubuntu 10.04 ou 12.04
	\item Java JDK 6 ou 7
	\item Android SDK
	\item Android 2.2 API 8
	\item Eclipse Juno
	\item ADT Plugin
	\item Sqlite3
	\item Sqliteman
	\item Inkscape
\end{itemize}

Você pode estar se perguntando: ''Por que utilizar essa configuração?''. Bom, para começar
um ambiente de desenvolvimento precisa ser estável, e para isso nada melhor que o
\url{http://releases.ubuntu.com/lucid/} (Ubuntu 10.04) ou ainda o \url{http://releases.ubuntu.com/precise/}
(Ubuntu 12.04) por serem \gls{lts}.

A \gls{ide} Eclipse funciona independente do sistema operacional, então podemos utilizar a versão
mais recente. O mesmo para o plugin \gls{adt}.

Usaremos especificamente para os exemplos a seguir a versão 2.2 do Android. Essa \gls{api} é uma
ótima escolha inicial, pois é a mais utilizada pelos aparelhos mais simples que rodam Android.
É claro que você poderá instalar outras versões e compilar seus aplicativos para \textit{tablets}, etc.

\section{Instalação}

A instalação do Java JDK 6 no Ubuntu 10.04 não é a mesma que no Ubuntu 12.04. Isso porque na época
de lançamento do \gls{lucid}, em 2010, a empresa que desenvolvia o Java era a Sun Microsystems, que
tinha um canal nos repositórios do Ubuntu como parceira (\textit{partner}). Ainda em 2010 a empresa
Oracle comprou a Sun junto com seu software e hardware. Nesse ponto o canal de parceria foi desligado.

Discussões a parte, vamos ver as duas maneiras de instalar o Java.

\subsection{Java JDK 6}

A instalação do Java no Ubuntu 10.04 é bastante simples. Você apenas irá precisar habilitar repositório
de terceiros, ou \textit{Partner}. Isso pode ser feito através do aplicativo \textit{Synaptic}. No menu
principal do Ubuntu clique em \texttt{Sistema $\rightarrow$ Administração $\rightarrow$ Gerenciador de pacotes
Synaptic}.

No menu do Synaptic clique em \texttt{Configuração $\rightarrow$ Repositórios}. Na aba \texttt{Outro Software}
temos vários itens que representam repositórios. Marque os dois repositórios que terminam com \texttt{partner}.
Feche e depois clique em \texttt{Editar $\rightarrow$ Recarregar informações dos pacotes} ou simplesmente
\texttt{Ctrl + R}.

Após a atualização dos pacotes existentes nos repositórios já é possível encontrar o Java \gls{jdk} 6.
No campo de \texttt{Pesquisa rápida} digite: \texttt{sun-java6}. Clique com botão direito no pacote
\texttt{sun-java6-jdk} e selecione a opção \texttt{Marcar para instalação}. Depois basta \texttt{Aplicar}
as mudanças. Para isso clique em \texttt{Editar $\rightarrow$ Aplicar as alterações marcadas} ou \texttt{Ctrl + P}.

\bigskip

Para a instalação no Ubuntu 12.04 temos que habilitar um repositório de terceiros, também conhecido como
\gls{ppa} (Personal Package Archives). Abra um terminal e execute os passos a seguir para adicionar um
repositório e instalar o Java:

\begin{flushleft}\texttt{\$ sudo su\\
\# apt-add-repository ppa:flexiondotorg/java\\
\# apt-get update\\
\# apt-get install sun-java6-jdk\\}
\end{flushleft}

\subsubsection{Um pouco de Linux}

Para quem não está familiarizado com o ambiente Linux vamos a uma pequena explicação. Nos comandos
acima aparecem dois caracteres que não devem ser escritos mas que representam algo importante no âmbito
dos comandos, são eles \texttt{\$} e \texttt{\#}. Estes caracteres indicam qual o nível do usuário;
\texttt{\$} significa usuário comum, \texttt{\#} representa super usuário (\texttt{root}). No comando
\texttt{sudo su} é onde trocamos de usuário comum para super usuário. Neste momento você terá que entrar
com sua senha de \textit{login}.

\subsubsection{Java JDK 7}

Segundo a página de Requerimentos do Sistema (\url{http://developer.android.com/sdk/requirements.html})
do site oficial do Android, é necessário uso do Java 6. Caso você queira utilizar o Java 7, você
terá que configurar seu projeto Android para ser compilado com suporte a versão 6.

A instalação no Ubuntu 12.04 pode ser feita da seguinte maneira:

\begin{flushleft}\texttt{\$ sudo su\\
\# add-apt-repository ppa:webupd8team/java\\
\# apt-get update\\
\# apt-get install oracle-jdk7-installer\\}
\end{flushleft}

Após criar o projeto clique com botão direito do \textit{mouse} em seu projeto e selecione
\texttt{Properties}. Na lista de itens do lado esquerdo selecione \texttt{Java Compiler}. Daí basta clicar
em \texttt{Enable project specific settings} e logo abaixo escolher o nível de compilação em
\texttt{Compiler compliance level}, escolha \texttt{1.6}.

\subsection{Android SDK \label{ssec:sdk}}

Para o Android \gls{sdk} comece pelo \textit{download} \url{http://developer.android.com/sdk/index.html}.

A instalação é feita apenas colocando o SDK em um diretório do sistema. Existem 2 bons locais para
abrigar bibliotecas no Linux, são elas: \texttt{/opt} e \texttt{/usr/local/lib}. Nesse exemplo vamos
utilizar este último. Abra um terminal e vamos aos comandos.

Se você baixou o SDK para seu diretório \textit{Downloads}, proceda da seguinte maneira:

\medskip

\begin{flushleft}
\texttt{\$ cd /home/usuario/Downloads \\
\$ tar -xf android-sdk\b{ }r18-linux.tgz \\
\$ sudo su \\
\# mv android-sdk-linux /usr/local/lib \\
\# cd /usr/local/lib \\
\# ln -s android-sdk-linux android-sdk \\
\# cd android-sdk/tools \\
\# ln -s android /usr/local/bin/android \\
}
\end{flushleft}

\paragraph{Obs.:} troque \texttt{usuario} na primeira linha pelo seu \textit{login} do sistema.

\subsubsection{O poder do Linux}

Dê atenção ao uso do comando \texttt{ln}. Ele é responsável por criar \textit{links} simbólicos. Isso é
muito útil quando se instala um aplicativo ou biblioteca, pois proporciona atualização sem que
outros ajustes sejam feitos. Neste caso basta \textit{linkar} outra vez e pronto.

Note que no último comando temos um link simbólico para o diretório \texttt{/usr/local/bin}. É
nele que colocamos os executáveis globais, ou seja, que são vistos a partir de qualquer outro
diretório. Agora saindo do modo \textit{root} e usando seu próprio usuário instalaremos a API.

\subsection{Android 2.2 API 8}

Ainda no terminal, agora como usuário comum, vamos abrir o aplicativo que instala qualquer uma
das API disponibilizadas pelo Android.

\medskip

\begin{flushleft}
\texttt{\$ android}
\end{flushleft}

\medskip

O aplicativo \texttt{Android SDK and AVD Manager} irá aparecer. Clique em \texttt{Avaliable packages}
e procure pela versão 2.2 API 8 do Android. Selecione e clique em \texttt{Install Selected}. Após
o download você pode verificar a versão instalada em \texttt{Installed packages}, um dos itens é
algo como \texttt{SDK Plataform Android 2.2, API 8, revision 2}.

Se você quiser aproveite para baixar outras versões para utilizar em projetos futuros.

\subsection{Android Virtual Device (AVD)}

Vamos aproveitar e criar nosso \gls{avd} para testar pela primeira vez nosso emulador. Ainda no
\texttt{Android SDK and AVD Manager} clique em \texttt{Virtual devices}, depois em \texttt{New...}

Dê um nome. Você pode usar qualquer nomenclatura, mas é interessante que tenha algo haver com a versão. Assim,
caso você tenha que testar seu código em outras versões você poderá saber qual emulador utilizar. Por
exemplo use \texttt{android-2.2}. Em \texttt{Target} escolha a versão, neste caso
\texttt{Android 2.2 - API Level 8}. Pronto, apenas clique em \texttt{Create AVD}.

\subsubsection{Dicas}

A opção \texttt{Skin} indica qual a resolução da tela do aparelho. Como não é possível
redimensionar a janela, em alguns monitores a janela fica maior que a tela do seu monitor.

A opção \texttt{Snapshot} quando habilitada, serve para salvar o estado do emulador. Isso faz
com que da segunda inicialização em diante se torne mais rápida.

A opção \texttt{SD Card} é ideal caso sua aplicação necessite guardar dados como fotos, arquivos.
O AVD irá reservar um espaço em seu HD permitindo assim o acesso a dados pelo emulador.

\subsection{Eclipse Juno}

O IDE Eclipse pode ser encontrada em \url{http://www.eclipse.org/downloads/}. Para o desenvolvimento
de aplicativos para o Android a versão \texttt{Eclipse IDE for Java Developers} é ideal. Mas se você
tiver interesse em aplicativos Java para Web a opção é baixar a versão \texttt{Eclipse IDE for Java EE Developers}.

Em todo caso as duas vão servir para o desenvolvimento, pois ambas vem com suporte a Java.

O Eclipse não possui instalador, no caso ele já vem pré-compilado. Basta apenas descompactar e executar
o arquivo \texttt{eclipse}.

Para sua comodidade você pode adicionar o Eclipse no menu do Ubuntu. Isso pode ser feito apenas clicando
com o botão direiro do \textit{mouse} no menu principal e escolhendo a opção \texttt{Editar menus}. Ou você pode
usar a dica do blog MAD3 Linux \\ (\url{http://www.mad3linux.org}) - \url{http://va.mu/VSgR}. Essa dica irá
lhe mostrar como adicionar um item ao menu visível a todos os usuários.

\subsection{Plugin ADT}

Para a instalação do plugin ADT vamos abrir o Eclipse, e em seu menu selecione \texttt{Help $\rightarrow$
Eclipse Marketplace...}

Busque por \texttt{adt} e escolha o \texttt{Android Development Tools for Eclipse} da \texttt{Google, Inc.,
Apache 2.0} e clique em \texttt{Install}. O Eclipse irá pedir confirmação sobre os itens a serem instalados,
clique em \texttt{Next}. Agora basta aceitar os termos de uso e clicar em \texttt{Finish}. Após o download
e a instalação, reinicie o Eclipse.

No \texttt{Eclipse Marketplace} você pode encontrar outras ferramentas bastante úteis para um bom desenvolvimento.
Clique na aba \texttt{Popular} e veja as ferramentas mais baixadas, talvez exista uma que você não conheça
mas que sempre precisou.

\subsubsection{Configurando o ADT}

Agora que o plugin foi instalado temos que dizer ao Eclipse onde nós instalamos o Android SDK. Isso
pode ser feito clicando no menu \texttt{Window $\rightarrow$ Preferences}. Selecione \texttt{Android} no painel
lateral esquerdo. Em \texttt{SDK Location} clique em \texttt{Browse...} e indique o diretório do SDK,
caso não lembre, ele está em \texttt{/usr/local/lib/android-sdk}. Clique em \texttt{Apply} na parte inferior
direita para atualizar a lista de API's disponíveis.

Caso você tenha mais dúvidas dê uma olhada na página oficial de instalação do plugin ADT localizada em
\url{http://developer.android.com/sdk/eclipse-adt.html}.

\subsubsection{Testando o ADT \label{sssec:testando}}

Para testar o \texttt{Android Development Tools} ou ADT crie um projeto Android. No menu do Eclipse selecione
\texttt{File $\rightarrow$ New $\rightarrow$ Project...}

Selecione \texttt{Android Application Project} e clique em \texttt{Next}. Dê um nome qualquer ao seu aplicativo,
por exemplo \texttt{hello.android}. Note que o ADT tenta dar um nome ao seu pacote e ao diretório de arquivos
a partir do nome que você digitou. Deixe como está. Em \texttt{Build SDK} é preciso escolher qual API vamos
utilizar, em nosso caso escolha a \texttt{Android 2.2 (API 8)}. Em \texttt{Minimum Required SDK} escolha
a \texttt{API 8: Android 2.2 (Froyo)} indicando que a versão mínima é a \texttt{API 8}. Clique em \texttt{Next}.

Na versão do ADT para o Eclipse Juno, uma novidade apareceu. É possível criar o ícone lançador logo ao criar
o aplicativo. Selecione da maneira que achar melhor e clique em \texttt{Next}. Depois clique em \texttt{Finish}.

Após isso clique com botão direito do \textit{mouse} no projeto recém criado, e \texttt{Run As $\rightarrow$
Android Application}. Se tudo tiver dado certo é possível ver no emulador sua primeira aplicação
rodando.

\subsubsection{Dicas}

Uma vez que você abriu o emulador não o feche. Você irá notar que ao abrir
pela primeira vez ele leva um tempo para isso. Neste caso ao atualizar o código-fonte apenas
rode o aplicativo novamente. O plugin ADT fará com que o aplicativo seja reinstalado no emulador.

Faça o teste com alguns atalhos básicos:
\begin{description}
	\item[Alt + Enter] Maximiza o emulador. Ideal para demostrações.
	\item[Ctrl + F11] Muda a orientação do emulador, retrato ou paisagem.
	\item[F8] Liga/desliga a rede.
\end{description}

Outro elemento essencial é o \texttt{LogCat}. Ele faz parte do ADT e é responsável por mostrar
as mensagens de \textit{log} do emulador. Caso você encontre problemas com seu código o \texttt{LogCat}
será seu melhor aliado. Para acessá-lo no Eclipse clique no menu \texttt{Window $\rightarrow$ Show View
$\rightarrow$ Other...}, clique em \texttt{Android $\rightarrow$ LogCat}.

\subsection{Sqlite3}

Sendo o Sqlite o banco de dados embutido na plataforma Android, nada melhor do que aprendermos um pouco
sobre ele.

O Sqlite é um banco de dados relacional bastante utilizado por dispositivos e sistemas embarcados por
ser leve, robusto, de fácil configuração e, acima de tudo, livre. Para a instalação, abra um terminal
como \texttt{root} e:

\begin{flushleft}
\texttt{\$ sudo su \\
\# apt-get install sqlite3 \\}
\end{flushleft}

Após a instalação é possível utilizar o Sqlite via linha de comando. Faça \textit{logoff} do usuário
\texttt{root} e faça os seguintes testes:

\begin{flushleft}
\texttt{\# exit \\
\$ sqlite \\
SQLite version 2.8.17\\
Enter ".help" for instructions\\
sqlite>\\}
\end{flushleft}

Você deverá ver algo parecido. Para sair utilize o comando \texttt{.exit}. Veja outros detalhes na página
oficial do projeto: \url{http://www.sqlite.org/}.

\subsubsection{Tipos de dados}

Utilize a tabela abaixo para criar suas tabelas futuramente.

\begin{table}[H]
\begin{tabularx}{400pt}{lX}
\hline
\textbf{Nome} & \textbf{Descrição} \\
\hline
\texttt{INTEGER} & valores inteiros, positivos ou negativos. Podem variar de 1 a 8 \textit{bytes}.\\
\texttt{REAL} & valores reais ou decimais.\\
\texttt{TEXT} & usado para armazenar valores, não-limitado. Suporta várias codificações, por exemplo \texttt{UTF-8}.\\
\texttt{BLOB} & objetos binários tais como imagens, arquivos de texto, etc. Também possui tamanho não-limitado.\\
\texttt{NULL} & representa falta de informação.\\
\hline
\end{tabularx}
\caption{Tipos de dados do Sqlite}
\end{table}

\subsection{Sqliteman}

Para uma gerência mais produtiva usaremos o Sqliteman para acessar e modificar bancos de dados. A instalação
é feita via linha de comando. Abra um terminal e:

\begin{flushleft}
\texttt{\$ sudo su\\
\# apt-get install sqliteman\\}
\end{flushleft}

Depois de instalado, acesse o aplicativo do menu principal do Ubuntu em \texttt{Aplicativos $\rightarrow$
Escritório $\rightarrow$ Sqliteman}. Faça alguns testes criando bancos de dados, depois crie algumas tabelas.
Ele possui assistentes que irão auxiliar nos primeiros momentos.

Por exemplo, crie uma base de dados e depois clique com o botão direito do \textit{mouse} em \texttt{Tables}.
Utilize o assistente e veja como é simples criar tabelas no sqlite.

% exemplo_bd.sql
\begin{listing}[H]
  \inputminted[linenos=true,frame=bottomline,tabsize=3]{ sql }{ source/exemplo-bd-1.sql }
  \caption{Exemplo de banco de dados [exemplo-bd.sql]}
\end{listing}

Observe que podemos fazer auto-relacionamento na tabela. Assim somos capazes de executar a seguinte SQL,
contando o número de distros que derivam de uma outra original. Veja:

% exemplo_bd.sql
\begin{listing}[H]
  \inputminted[linenos=true,frame=bottomline,tabsize=3]{ sql }{ source/exemplo-bd-2.sql }
  \caption{Exemplo de \textit{query} com \textit{subquery} [exemplo-bd.sql]}
\end{listing}

Mais informações em: \url{http://sqliteman.com/}

\subsection{Inkscape}

Uma ótima ferramenta de desenho vetorial é o Inkscape. Ela será bastante útil pois o desenvolvimento
de aplicativos hoje em dia é baseado muito em figuras para facilitar a navegação, identidade visual,
entre outras coisas.

A instalação é feita de forma simples. Num terminal:

\begin{flushleft}
\texttt{\$ sudo su\\
\# apt-get install inkscape\\}
\end{flushleft}

Para dicas de como criar ícones para os diversos elementos do Android veja a página
\url{http://developer.android.com/design/style/iconography.html}.
